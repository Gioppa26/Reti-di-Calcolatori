
\chapter{Chapter 1}\label{ch:Chapter1}
\section{Common features of networks}
Una rete è composta da:
\begin{itemize}
  \item Nodi: possono essere computer, utenti,etc.
  \item Informazioni
  \item Canali di comunicazione: possono essere canali elettrici,ottici, chimici, virtuali, etc.
  \item Tasks
\end{itemize}
\section{Features delle reti}
\subsection{General purpose}
\begin{itemize}
  \item non ottimizzate per una spefigica applicazione
  \item permette a più applicazioni di coesistere nella stessa rete
\end{itemize}
\subsection{Openness,flexibility}
\begin{itemize}
  \item Possibilità di aggiungere nuove feature in modo dinamico
\end{itemize}
\begin{example}[World Wide Web]
 E' uno dei servizi principale di internet
  \begin{itemize}
    \item URL: Uniform Recource Locater
    \item HTTP: Hyper Text Transfer Protocol
    \item TPC: Transmission Control Protocol
  \end{itemize}
\end{example}
\section{Requisiti}
\begin{itemize}
  \item Application Programmer: cosa vuole nella sua applicazione
  \item Progettista della rete: cerca di sfruttare al massimo le risorse avendo un budget ridotto.
  \item Network provider: elenca le caratteristiche di un sistema che sono facilmente facili da mantenere.
\end{itemize}
\section{Connettività}
